\documentclass{article}
\usepackage{amsmath, amsthm, amsfonts, amssymb, geometry, natbib}

% Set margin to 2.5cm on all sides
\geometry{margin=2.5cm}

\title{Occult Mathematical Philosophy}
\author{Henry Cornelius Agrippa, Knight}
\date{2021}

\begin{document}

\maketitle

\tableofcontents

\section{Introduction}
This outlines the four elements, astrology, Kabbalah, numerology, angels, names of God, the virtues and relationships with each other as well as methods of utilizing these relationships and laws in medicine, scrying, alchemy, numerology, ceremonial magic, origins of what are from the Hebrew, Greek and Chaldean context.

\subsection{Formatting Text}
Here are examples of formatting text (Bold) and (Italic):
\begin{itemize}
    \item \textbf{Example of Bold Text}: \textbf{Mathematical Philosophy teacheth us to know the quantity naturall Bodies, as extended into three dimensions, as also to conceive of the motion,
    and the course of Celestial Bodies}
    \item \textit{Example of Italic Text}: \textit{Hence by the Heavens we may foreknow, the seasons all times for to reap and sow, and when 'tis fit to launch into the deep, and when to war, and when in peace to sleep, and when to dig up trees, annd them again, to set that so they may bring forth amain.}
\end{itemize}

\section{Mathematical Notation}
Here are some examples of mathematical equations:
\begin{equation}
    E = mc^2
\end{equation}

\begin{equation}
    \int_{a}^{b} x^2 \, dx = \frac{b^3 - a^3}{3}
\end{equation}

\begin{equation}
    f(x) = \sum_{n=0}^{\infty} \frac{f^{(n)}(a)}{n!}(x - a)^n
\end{equation}

\begin{equation}
    \vec{F} = m \vec{a}
\end {equation} 

\section{BibTeX}
Here is a reference using BibTeX \citep{OccultPhilosophy}.

\section{Table Example}
\begin{table}[h]
    \centering
    \begin{tabular}{|c|c|c|}
    \hline
    \textbf{Title} & \textbf{Meaning} & \textbf{Description} \\
    \hline
    In the original world & The Father & The son, Sadai \\
    In the intellectual world & Supreme, Innocents & Succeeding \\
    In the elementary world & Simple & Compounded \\
    \hline
    \end{tabular}
    \caption{Example Table of three books of occult philosophy pg. 104}
    \label{tab:example}
\end{table}

\section{Font Families}
\subsection{Serif}
The Pythagorians call the number four "tetractis", and prefer it before all the virtues of numbers, because it is the foundation and root of all other numbers.

\subsection{Sans-serif}
\textsf{The number five is of no small force, for it consists of the first even, and the first odd, as of a Female and Male, for an odd number is Male and the even number is Female.}

\subsection{Mono-spaced}
\texttt{Six is the number of perfection, because it is the most perfect in nature, in the whole course of numbers, from one to ten, and it alone is so perfect, that in the collection of parts it results the same, neither wanting nor abounding.}

\section{Rulers}
\noindent\makebox[\linewidth]{\rule{\textwidth}{0.4pt}}

\section{Spaces}
\subsection{Horizontal Spaces}
Also there are four rivers of paradise. \quad Four gospels received from four evangelists throughout the whole church. \\
The Egyptians, Arabians, Persians, Magicians, Mahumitans, Latines, Grecians, Tuscans . \qquad Write the name of God with only four letters.

\subsection{Vertical Spaces}
The name of God with four letters.\\[6pt]
Four angels ruling over the coners of the world. 

\section{Theorems and Proofs}
\subsection{Pythagorean Theorem}
\begin{theorem}[Pythagorean Theorem]
In a right triangle, the square of the length of the hypotenuse is equal to the sum of the squares of the lengths of the legs. 
\end{theorem}

\begin{proof}
Let \( a \), \( b \), and \( c \) be the lengths of the sides of a right triangle. Without loss of generality, assume \( c \) is the length of the hypotenuse. Then, by the Pythagorean theorem, we have:
\[
a^2 + b^2 = c^2
\]
\end {proof}

\subsection{Alternative Proof of the Pythagorean Theorem}
\begin{theorem} 
Consider a right triangle with sides of length $a$, $b$, and $c$, and right angle at $C$. Using the Law of Cosines, we have:
\end{theorem}
\begin{proof}

    c^2 &= a^2 + b^2 - 2ab\cos(\theta) \\
    &= a^2 + b^2 - 2ab(0) \\
    &= a^2 + b^2

\end{proof} 


\bibliographystyle{plainnat}
\bibliography{references}

\end{document}
